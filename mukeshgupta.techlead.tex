%% The MIT License (MIT)
%%
%% Copyright (c) 2015 Daniil Belyakov
%%
%% Permission is hereby granted, free of charge, to any person obtaining a copy
%% of this software and associated documentation files (the "Software"), to deal
%% in the Software without restriction, including without limitation the rights
%% to use, copy, modify, merge, publish, distribute, sublicense, and/or sell
%% copies of the Software, and to permit persons to whom the Software is
%% furnished to do so, subject to the following conditions:
%%
%% The above copyright notice and this permission notice shall be included in all
%% copies or substantial portions of the Software.
%%
%% THE SOFTWARE IS PROVIDED "AS IS", WITHOUT WARRANTY OF ANY KIND, EXPRESS OR
%% IMPLIED, INCLUDING BUT NOT LIMITED TO THE WARRANTIES OF MERCHANTABILITY,
%% FITNESS FOR A PARTICULAR PURPOSE AND NONINFRINGEMENT. IN NO EVENT SHALL THE
%% AUTHORS OR COPYRIGHT HOLDERS BE LIABLE FOR ANY CLAIM, DAMAGES OR OTHER
%% LIABILITY, WHETHER IN AN ACTION OF CONTRACT, TORT OR OTHERWISE, ARISING FROM,
%% OUT OF OR IN CONNECTION WITH THE SOFTWARE OR THE USE OR OTHER DEALINGS IN THE
%% SOFTWARE.

% The font could be set to Windows-specific Calibri by using the 'calibri' option
\documentclass[]{mukeshcv}

% For mathematical symbols
\usepackage{amsmath}

% Set applicant's personal data for header
\name{Mukesh Gupta}
\address{2/189, RHB Colony \linebreak Bhiwadi, Alwar, Rajasthan 301019}
\contacts{(+91) 956-040-7662 \linebreak mukeshgupta.2006@gmail.com}

\begin{document}\thispagestyle{empty}


	% Print the header
	\makeheader
	
	\begin{cvsection}{About}
		\begin{cvsubsection}{}{}{}
			I'm an engineer with 4.5+ years of experience in Desktop \& Web application development. During my 2.5 years of entreprenurial journey, I learned how to start \& operate a SaaS business and scaled it from 0 to 300 paying customers. I worked on different aspects of web application development from conceptualization and planning to development, testing, deployment and scaling. As a developer, I believe in writing code which is DRY, modular and maintainable.
		\end{cvsubsection}
	\end{cvsection}

	\begin{cvsection}{Technical Skills and Expertise}
		\begin{cvsubsection}{}{}{}	
			\begin{itemize}
				\item \textbf{Expertise}: Backend Development, Architecture Design, API Design, Deployment
				\item \textbf{Languages}: Python, JavaScript, C++, C, SQL
				\item \textbf{Frameworks}: Django, Angularjs, Nodejs, Mochajs, Jasmine
				\item \textbf{Infrastructure}: Docker, AWS(EC2, S3, RDS, SES)
				\item \textbf{Technologies}: Celery, MySQL, Git, Fabric(Python), Grunt, Apache, Redis, Haproxy

			\end{itemize}
		\end{cvsubsection}
	\end{cvsection}

	% Print the content
	\begin{cvsection}{Employment}
		\begin{cvsubsection}{Co-Founder \& Tech lead}{Scanova}{October 2013 -- March 2016}
		Scanova is an online QR Code Management platform that helps businesses acquire leads through their offline print marketing campaigns. As a Co-Founder, I was routinely involved in strategic planning, architecture design \& development of our tech stack and product planning.  \\

			\begin{itemize}
				\item \projecttitle{Analytics Engine}: Designed \& Developed Scanova Analytics Engine to generate the QR code Scan Analytics. The information was extracted from browser's user-agent and client IP address and stored in a MySQL database. A RESTful API was implemented to retrieve the Analytics.
				\item \projecttitle{Designer QR Code API}: Developed a  designer QR Code generation API in Nodejs. Used Token Authentication Scheme for user authentication, identification and rate-limiting.
				\item \projecttitle{Scannability Prediction of colorful QR Codes}: Developed a predictive algorithm to predict the scannability of a designer QR code based on the colors used in the foreground and background. The algorithm predicts the cut-off foreground luminosity value for given hue \& saturation values of foreground and luminosity of background of QR Code. The technology was used to prevent the creation of un-scannable QR Codes.
				\item \projecttitle{Load Testing Framework for ExpressJs}: Built a framework for quick load testing of expressjs applications. The framework could be used with testing frameworks like jasmine to write bench-marking test cases for API response times.
				\item \projecttitle{Deployment Architecture}: Built a one-step deployment Architecture for complete application stack on AWS.
				\item \projecttitle{Dockerization}: Worked on dockerizing the QR Code Generation API as a part of making infrastructure scalable using microservices architecture.
			\end{itemize}
		\end{cvsubsection}
		\begin{cvsubsection}{Member of Technical Staff - II}{Adobe Systems}{June 2012 -- October 2013}	
			Adobe Illustrator
			\begin{itemize}
				\item \projecttitle{Touch type tool}: Designed and developed the touch type tool in Adobe Illustrator to modify the attributes of a character like kerning, baseline shift, font size, vertical scaling, and horizontal scaling in a text object using touch. (Released in Adobe Illustrator CS7).
				\item \projecttitle{Touch version of Free Transform Tool}: Designed and developed the touch based version of the Adobe Illustrator Free Transform tool. The project included implementing the architecture for routing touch and pen events to the illustrator core application framework and thereafter writing a UI Widget that would receive the events and map them to the geometric transformations. (Released in Adobe Illustrator CS7).
				\item \projecttitle{Multi-touch Support for UI Framework}: Collaboratively worked on adding the Multi-Touch support to internal UI framework. The framework was able to handle Multi-touch events from direct touch devices like Wacom Tablets, Windows touch-screen and indirect touch devices like Track pads.\\
			\end{itemize}
		\end{cvsubsection}
		\begin{cvsubsection}{Member of Technical Staff - I}{Adobe Systems}{June 2011 -- June 2012}
			Adobe Illustrator		
			\begin{itemize}
				\item Added MENA and Indic Language Support in Illustrator CS6.
				\item Collaboratively ported Adobe Illustrator CS6 to a new UI framework.
			\end{itemize}
		\end{cvsubsection}
		
		\begin{cvsubsection}{Software Developer, Intern}{One Laptop Per Child}{November 2010 -- January 2011}
			Sugar Chat Activity	
			\begin{itemize}
				\item Added smiley support to the Sugar Chat Application. \href{http://lists.sugarlabs.org/archive/sugar-devel/2010-December/029329.html}{Ref}
			\end{itemize}
		\end{cvsubsection}

		\begin{cvsubsection}{Software Developer, Intern}{STMicroelectronics}{May 2010 -- June 2010}
			\begin{itemize}
				\item Developed a software to identify patterns in the gate-level netlist of an application-specific integrated circuit (ASIC) using a directed graph pattern matching algorithm implemented in C++ using Boost C++ Libraries. The goal was to optimize the performance of the ASIC by adding the commonly occurring gate patterns to the VLSI standard cell library. 
			\end{itemize}
		\end{cvsubsection}
	\end{cvsection}
	
	\begin{cvsection}{Education}
		\begin{cvsubsection}{Delhi, India}{Delhi College of Engineering}{August 2007 -- May 2011}
			\begin{itemize}
				\item B.E. in Electronics and Communication. Aggregate percentage of marks: 76.3\% (First class with distinction).
				\item Head of technical affairs, IEEE student chapter, DCE, 2010-2011. Responsible for organizing technical activities and providing technical guidance to the organizing teams.	
			\end{itemize}
		\end{cvsubsection}
	\end{cvsection}
\end{document}
